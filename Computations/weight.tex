\documentclass[../main.tex]{subfiles}
\begin{document}
\subsection{Weights, nearby and vanishing cycles}
\begin{example}[$X=\mathbb{C}^{\times}\hookrightarrow\mathbb{P}_{\mathbb{C}}^{1}=\overline{X}$, weight filtration]
$D=\overline{X}\setminus X=\{*, *\}$. The weight spectral sequence is given by 
$$
E_{1}^{p,q}=\begin{tikzcd}2 \arrow{r}{d_{1}}& 1\\
0 & 0\\
0 & 1\end{tikzcd}\Rightarrow E_{2}^{p,q}=Gr_{q}H^{p+q}(X)=\begin{tikzcd}1 & 0\\
0 & 0\\
0 & 1\end{tikzcd}.$$
$d_{1}$ is surjective is because we know $E_{2}^{p,q}$ computes $H^{p+q}(X)$, which is just the cohomology of $S^{1}$. The filtration of $H^{1}(X)$ is given by 
$$
\begin{tikzcd}
0=W_{0}H^{1}(X)\subset 0=W_{1}H^{1}(X) \subset W_{2}H^{1}(X)=H^{1}(X; \mathbb{Q})=\mathbb{Q}\\
Gr_{2}^{W}H^{1}(X; \mathbb{Q})=W_{2}H^{1}(X;\mathbb{Q})/W_{1}H^{1}(X; \mathbb{Q})\cong \mathbb{Q}.
\end{tikzcd}
$$
\end{example}

\begin{remark}[Weight filtration, \href{http://userpage.fu-berlin.de/jcirici/computeweight.pdf}{reference, Joana Cirici
}]
Let $X$ be a smooth complex variety of dimension $n$. Let $j:X\hookrightarrow \overline{X}$ be a smooth compatification of $X$ such that $D=\overline{X}\setminus X$ is a normal crossing divisor. $D=\bigcup_{i=1}^{N}D_{i}$ is the decomposition of $D$ into irreducible components. Let $D^{(0)}=\overline{X}$, $D^{(p)}=\{\amalg_{\{i_{1}, \dots, i_{p}\}\subset\{1,2,\dots, N\}}D_{i_{1}}\cap \dots \cap D_{i_{p}}\}$, note that since $D$ is a normal crossing divisor, $D^{(p)}$ is a smooth projective variety of dimension $n-p$. The weight spectral sequence is given by 
$$E_{1}^{-p,q}=H^{q-2p}(D^{(p)};\mathbb{Q})\Rightarrow H^{q-p}(X;\mathbb{Q}).$$
The differential on the $E_{1}$-page is given by the sum of Gysin maps
$$d_{1}:E_{1}^{-p,q}\rightarrow E_{1}^{-p+1,q}$$
$$i_{*}(j): H^{q-2p}(D_{i_{1}}\cap \dots \cap D_{i_{p}})\rightarrow H^{q-2p+2}(D_{i_{i}}\cap \dots \hat{D}_{i}_{k}\dots \cap D_{i_{p}})$$, where $i$ is just the natural embedding. The weight spectral sequence degenerates on the $E_{2}$-page which gives us a filtration of the form 
$$0=W_{n-1}H^{n}(X; \mathbb{Q})\subset W_{n}H^{n}(X; \mathbb{Q})\subset \dots \subset W_{2n}H^{n}(X; \mathbb{Q})=H^{n}(X; \mathbb{Q}),$$
where $Gr_{q}^{W}H^{p+q}=E_{2}^{p,q}$ and $W_{k+1}H^{n}(X;\mathbb{Q})/W_{k}H^{n}(X)=Gr_{k+1}^{W}H^{n}(X)$. To make this discussion more obvious, see the following diagrams.
$$E_{1}^{-p,q}(X)=H^{q-2p}(D^{(p)}; \mathbb{Q})=
\begin{tikzcd}
\dots & \dots & \dots &  \\
H^{(1)}(D^{(2)}) \arrow{r}{d_{1}}& H^{3}(D^{(1)})\arrow{r}{d_{1}} & H^{5}(\overline{X}) & 5\\
H^{0}(D^{2})\arrow{r}{d_{1}} & H^{2}(D^{(1)}) \arrow{r}{d_{1}}& H^{4}(\overline{X}) & 4\\
0 & H^{1}(D^{(1)})\arrow{r}{d_{1}} & H^{3}(\overline{X}) & 3\\
0 & H^{0}(D^{(1)}) \arrow{r}{d_{1}}& H^{2}(\overline{X})& 2\\
0 & 0 & H^{1}(\overline{X}) & 1\\
0 & 0 & H^{0}(\overline{X}) & q=0\\
D^{(2)} & D^{1} & D^{(p=0)}=\overline{X} & \\
\end{tikzcd}
$$
$$
\Rightarrow E_{2}^{p,q}(X)=Gr_{q}H^{p+q}(X)
\begin{tikzcd}
\dots & \dots & \dots &  \\
\colorbox{pink}{$Gr_{5}^{W}H^{3}(X)$}&\colorbox{purple}{$Gr_{5}^{W}H^{4}(X)$} & \colorbox{gray}{$Gr_{5}^{W}H^{5}(X)$} & H^{5}\\
\colorbox{blue}{$Gr_{4}^{W}H^{2}(X)$} & \colorbox{pink}{$Gr_{4}^{W}H^{3}(X)$}& \colorbox{purple}{$Gr_{4}^{W}H^{4}(X)$} & H^{4}\\
\colorbox{yellow}{$0$} & \colorbox{blue}{$Gr_{3}^{W}H^{2}(X)$} & \colorbox{pink}{$Gr_{3}^{W}H^{3}(X)$} & H^{3}\\
\colorbox{red}{$0$} & \colorbox{yellow}{$Gr_{2}^{W}H^{1}(X)$} & \colorbox{blue}{$Gr_{2}^{W}H^{2}(X)$}& H^{2}\\
0 & \colorbox{red}{$0$} & \colorbox{yellow}{$Gr_{1}^{W}H^{1}(X)$} & H^{1}\\
0 &  0 & \colorbox{red}{$Gr_{0}^{W}H^{0}(X)$} & H^{q=0}\\
2 & 1 & p=0 & \\
\end{tikzcd}$$

\end{remark}
\begin{remark}[Gysin map]

\end{remark}

\begin{example}[punctured Riemann surface]
Let $\overline{X}$ be a Riemann surface of genus $g$. $D$ is given by $p$ distinct points, $X=\overline{X}\setminus D$. Then we have 
$$
E_{1}^{p,q}=\begin{tikzcd}p \arrow{r}{d_{1}}& 1\\
0 & 2g\\
0 & 1\end{tikzcd}\Rightarrow E_{2}^{p,q}=Gr_{q}H^{p+q}(X)=\begin{tikzcd}p-1 & 0\\
0 & 2g\\
0 & 1\end{tikzcd}.$$
Note that we don't need to do any computation, since we know $X$ is homotopic to some wedge product of several $S^{1}$'s, thus $H^{2}(X;\mathbb{Q})$ and above all vanish. This gives us the surjectivity of $d_{1}$. The only interesting part is the filtration of $H^{1}(X;\mathbb{Q})$ which is given by 
$$0=W_{0}H^{1}(X;\mathbb{Q})\subset W_{1}H^{1}(X; \mathbb{Q})\cong \mathbb{Q}^{2g}\subset W_{2}H^{1}(X;\mathbb{Q})\cong H^{1}(X;\mathbb{Q})\cong \mathbb{Q}^{2g+p-1},$$
$$Gr_{1}H^{1}(X;\mathbb{Q})\cong \mathbb{Q}^{2g}, Gr_{2}H^{1}(X;\mathbb{Q})\cong \mathbb{Q}^{p-1}.$$
\end{example}
\begin{example}[complement of three intersecting lines in $\mathbb{P}^{2}$]
Let $\overline{X}=\mathbb{P}^{2}$, $D$ be the union of three lines intersecting at $3$ different points, which is of course a normal crossing divisor. $X=\overline{X}\setminus D$. Now we have $D^{(1)}=\{\mathbb{P}^{1},\mathbb{P}^{1},\mathbb{P}^{1}\}, D^{(2)}=\{*, * ,*\}$, run the weight spectral sequence we get  
$$
E_{1}^{p,q}=\begin{tikzcd}
3 \arrow{r}{d_{1}}&3\arrow{r}{d_{1}} & 1\\
0 &0 & 0\\
0 &3 \arrow{r}{d_{1}}& 1\\
0 &0 & 0\\
0 &0 & 1\end{tikzcd}\Rightarrow E_{2}^{p,q}=Gr_{q}H^{p+q}(X)=\begin{tikzcd}
1 &0 & 0\\
0 &0 & 0\\
0 &2 & 0\\
0 &0 & 0\\
0 &0 & 1\end{tikzcd}.$$
Note again, here we don't need to do any computation, topologically, $\mathbb{P}^{2}\setminus\mathbb{P}^{1}\cong \mathbb{R}^{4}$, then we need to remove two more intersecting lines in $\mathbb{R}^{2}\times \mathbb{R}^{2}$, for example, let's just remove $0\times \mathbb{R}^{2}$ and $\mathbb{R}^{2}\times 0$, this tells us $X$ is homotopic to $S^{1}\times S^{1}$. Then the $E_{2}$-page above is the only possibility. And it's easy to describe the filtration.
\end{example}
\begin{example}[complement of three concurrent lines in $\mathbb{P}^{2}$]
The point of this example is that three concurrent lines in $\mathbb{P}^{2}$ is not a normal crossing divisor. However, it's not a problem at all, we can blow them up w.r.t to the intersecting point , let's denote the blow up by $\overline{X}=Bl_{p}(\mathbb{P}^{2})$. Set $D$ to be $\{L_{1}, L_{2}, L_{3}, E\}$, where $E$ is the exceptional divisor. Then $\overline{X}\setminus D$ is isomorphic to the complement of three concurrent lines. Now $D^{(1)}=\{\mathbb{P}^{1},\mathbb{P}^{1},\mathbb{P}^{1},\mathbb{P}^{1}\}$, $D^{(1)}=\{*, *, *\}$. Apply the weight spectral sequence, we get 
$$
E_{1}^{p,q}=\begin{tikzcd}
3 \arrow{r}{d_{1}}&4\arrow{r}{d_{1}} & 1\\
0 &0 & 0\\
0 &4 \arrow{r}{d_{1}}& 2\\
0 &0 & 0\\
0 &0 & 1\end{tikzcd}\Rightarrow E_{2}^{p,q}=Gr_{q}H^{p+q}(X)=\begin{tikzcd}
0 &0 & 0\\
0 &0 & 0\\
0 &2 & 0\\
0 &0 & 0\\
0 &0 & 1\end{tikzcd}.$$
First, notice that the $2$ in the $E_{1}$-page is because we have an extra $\mathbb{P}^{1}$ given by the exceptional divisor. Then we know $Gr_{2}H^{1}(X;\mathbb{Q})\cong \mathbb{Q}^{2}$. Note again, by a similar argument as above $X$ is homotopic to $S^{1}\vee S^{1}$.
\end{example}
\begin{remark}[$X=\mathbb{C}^{2}\setminus\{0\}\subset \mathbb{P}^{2}$]
Use the same method again(we mean, blow-up), it's not hard to see the weight filtration of $H^{n}(X;\mathbb{Q})$.
\end{remark}

\begin{example}[cubical hyperresolution, It's our guess, we didn't check the precise definition carefully!!!!!!]
$$
\begin{tikzcd}
& \colorbox{yellow}{$*$}\arrow{rr}\arrow{dd}& &\framebox(9,9){$C'$}\arrow{rr}\arrow{dd} & & \colorbox{blue}{$X$}\arrow{dd}\\
\colorbox{red}{$*, * $}\arrow{rr}\arrow{dd}\arrow{ur}& &\colorbox{yellow}{$\mathbb{P}^{1}$}\arrow{ur}\arrow{dd} & & &\\
& \colorbox{blue}{$*$}\arrow{rr}&  & \framebox(9,9){$*$}\arrow{rr} & & \framebox(9,9){$Y$}\\
\colorbox{yellow}{$*$} \arrow{rr}\arrow{ur}& &\colorbox{blue}{$*$} \arrow{ur} & & &\\
\end{tikzcd}
$$
\end{example}

\begin{example}[Nodal curve]
Consider $X=\{(x,y,z)\subset \mathbb{P}^{2}|y^{2}z=x^{2}(x+z)\}$, then the standard resolution is given by 
$$\begin{tikzcd}
\tilde{Y}=\{*, *\}\arrow{r}\arrow{d} & \widetilde{X}\cong \mathbb{P}^{1}\arrow{d}{f}\\
Y=\{*=(0,0)\}\arrow{r} & X\end{tikzcd}$$
\end{example}

\begin{remark}[weight spectral sequence for singular varieties]

\end{remark}

\begin{remark}[Mayer-Vietoris exact sequence]

\end{remark}
\begin{remark}[cubical hyperresolution]
\href{https://www.math.utah.edu/~schwede/Notes/NotesOnCubicalHyperresolutions.pdf}{cubical hyperresolution}
\href{https://link.springer.com/content/pdf/10.1007/978-3-319-28829-1_15.pdf}{Some Remarks on Hyperresolutions, J.H.M. Steenbrink}
\end{remark}

\begin{example}[cuspidal singularity]
\end{example}
\begin{example}[de Rham-Witt complex]
See the following 
\begin{itemize}
\item \href{https://www3.nd.edu/~mbehren1/TAGS/Davis_notes.pdf}{de Rham-Witt complex}
\end{itemize}
\end{example}

\subsection{Nearby cycles}
\begin{example}[Tate twist]
\href{http://www.math.sci.hiroshima-u.ac.jp/algebra/agsympo/documents/Sato.pdf}{$p$-adic Tate twist}
\end{example}
\begin{example}[inertia group]
\href{https://en.wikipedia.org/wiki/Ramification_group}{Ramification group}
\end{example}



\begin{example}[Frobenius morphisms]

\end{example}
\begin{example}[monodromy operators]
\href{http://achinger.impan.pl/thesis.pdf}{page 14}
\end{example}

\begin{example}[Milnor fibration, nearby cycles of a family $X\rightarrow D$, \href{http://achinger.impan.pl/thesis.pdf}{reference}]
The definition of Milnor fibre exactly means close enough to a point $x\in X_{0}$, the family over some small  punctured disk $S^{*}$ is trivial, then just choose one typical fibre in this family. And the classical definition or the definition  by a base change to the the universal covering of $S^{*}$(\href{http://achinger.impan.pl/thesis.pdf}{see here, page $3$}) is just a way to make out choice in some sense `canonical' or `unambiguous'.   Consider the family 
$$X=\{(x_{1}, x_{2}, t)|x_{1}x_{2}=t, |t|<1\}\rightarrow D$$
$$(x_{1}, x_{2}, t)\mapsto t.$$
Then by the intuition of `nearby cycle', we know the Milnor fibre over $pt=(0,0,0)$ is given by $x_{1}x_{2}=\epsilon\neq 0$, which is just $(a, \frac{1}{a})$, which is just $\mathbb{C}^{\times}$.
This intuition can be verified by the definition, which say that the Milnor fibre over $pt$ is given by 
$$F_{pt}=\{(x_{1}, x_{2}, u)||x_{1}|^{2}+|x_{2}|^{2}+|x_{1}x_{2}|^{2}<\epsilon, \mathrm{exp}(u)=x_{1}x_{2}\}.$$
Note that the domain of $u$ is contractible, so $F_{pt}$ is a trivial $S^{1}$ fibration over a contractible base. We conclude that 
$$F_{p}\cong S^{1}.$$
And the sheaves of nearby cycles are just skyscraper sheaves supported at the point $pt$.
$$R\Psi\mathbb{Z}\cong \mathbb{Z}, R^{1}\Psi\mathbb{Z}\cong \mathbb{Z}_{pt}.$$
\end{example}

\begin{example}[Dwork family of elliptic curves, need 'monodromy formula']
Consider the family 
$$f: X=\{(\epsilon, [x,y,z]\in D\times \mathbb{P}_{\mathbb{C}}^{2}|\epsilon(x^{3}+y^{3}+z^{3})=3xyz\}\rightarrow D.$$
$$X_{0}$$ consists of three lines intersecting at $3$ different points. 
$$P_{1}=[1,0,0], P_{2}=[0,1,0], P_{3}=[0,0,1].$$
Note that $R^{0}\Psi\mathbb{Z}\cong \mathbb{Z}, R^{1}\Psi\mathbb{Z}\cong \oplus_{i=1}^{3}\mathbb{Z}_{P_{i}}$ and higher nearby cycles sheaves vanish because $R^{1}\Psi\mathbb{Z}$ is locally of rank 1. then the $E_{2}$-page of the nearby cycles spectral sequence looks like 
$$
\begin{tikzcd}
H^{0}(X_{0}, R^{1}\Psi\mathbb{Z})\arrow{drr} & H^{1}(X_{0}, R^{1}\Psi\mathbb{Z}) & H^{2}(X_{0}, R^{1}\Psi\mathbb{Z})\\
H^{0}(X_{0}, \mathbb{Z}) & H^{1}(X_{0}, \mathbb{Z}) & H^{2}(X_{0}, \mathbb{Z})
\end{tikzcd}
$$
$$
=\begin{tikzcd}
 \mathbb{Z}^{3}\arrow{drr}{d_{2}} & 0 & 0\\
\mathbb{Z} & \mathbb{Z} & \mathbb{Z}^{3}
\end{tikzcd}
$$
Since it degenerates to $H^{i}(X_{t}, \mathbb{Z})$, we know $\mathrm{ker}(d_{2})\cong \mathbb{Z}$. Or we can state this fact as the following exact sequence 
$$\begin{tikzcd}
0\arrow{r} & H^{1}(X_{0}, \mathbb{Z})\arrow{r}{i}\arrow{d}{=} & H^{1}(X_{t}, \mathbb{Z})\arrow{r}{p}\arrow{d}{=}&  H^{0}(X_{0}, R^{1}\Psi\mathbb{Z})\arrow{r}{d_{2}}\arrow{d}{=} & H^{2}(X_{0}, \mathbb{Z})\arrow{r}\arrow{d}{=}&  H^{2}(X_{t}, \mathbb{Z})\arrow{r}\arrow{d}{=}& 0\\
0\arrow{r} & \mathbb{Z}\arrow{r}{i} & \mathbb{Z}^{2}\arrow{r}{p}& \mathbb{Z}^{3}\arrow{r}{d_{2}} & \mathbb{Z}^{3}\arrow{r}&  \mathbb{Z}\arrow{r}& 0
\end{tikzcd}$$
We want to compute the monodromy action on $H^{1}(X_{t}, \mathbb{Z})$. By the construction in the reference, we know the monodromy action of $\pi_{1}(S^{*})$ on $R^{k}\Psi\mathbb{Z}$ is trivial if $i\geq 1$. Then as a $\pi_{1}(S^{*})$-module, 
Since the monodromy action doesn't affect the cohomology of the special fibre, we can choose a basis of $H^{1}(X_{t}, \mathbb{Z})=\{e_{1}, e_{2}\}$, where $e_{1}$ is the image of a generator of $H^{1}(X_{0}, \mathbb{Z})$, and $e_{2}$ is the pre-image of a generator of $\mathrm{ker}(d_{2})$, then the monodromy action of a generator $T$ of $\pi_{1}(S^{*})$ can be written in a form
$$T=\begin{pmatrix}1 & m\\
0 & 1\end{pmatrix}.$$
We want to figure out the integer number $m$ above. Apply the \href{http://achinger.impan.pl/thesis.pdf}{'monodromy formula'}, we know the dollowing diagram commutes up to sign. 
$$\begin{tikzcd}
H^{1}(X_{t}, \mathbb{Z}) \arrow{r}{p}\arrow{d}{1-T} &H^{0}(X_{0}, R^{1}\Psi\mathbb{Z})\arrow{d}{\overline{\alpha}}\\
H^{1}(X_{t}, \mathbb{Z}) & H^{1}(X_{0}, \mathbb{Z})\arrow{l}
\end{tikzcd}$$
where $\overline{\alpha}$ comes from the LES associated to the short exact sequence 
$$0\rightarrow\mathbb{Z}\rightarrow  \oplus_{i=1}^{3}\mathbb{Z}_{L_{i}}\rightarrow R^{1}\Psi\mathbb{Z}\cong\oplus_{i=1}^{3}\mathbb{Z}_{P_{i}}\rightarrow 0.$$
Use \Cech cohomology, we can compute 
$$\overline{\alpha}: H^{0}(X_{0}, R^{1}\Psi\mathbb{Z})\rightarrow H^{1}(X, \mathbb{Z})$$
$$[P_{i}]\mapsto 1.$$ 
The point is that $\mathrm{im}(p)=\mathrm{ker}(d_{2})$ is $\mathbb{Z}/3\mathbb{Z}$-invariant. Because let $\mathbb{Z}/3\mathbb{Z}$ act on $X$ by permuting the coordinates, then $p$ is naturally invariant almost by definition of the nearby cycle spectral sequence. And the we know $\mathrm{coker}(p)\cong \mathbb{Z}^{2}$ is torsion free, thus $\mathrm{im}(p)$ is generated by $[P_{1}]+[P_{2}]+[P_{3}]$, there for $\overline{\alpha}([P_{1}]+[P_{2}]+[P_{3}])=3$. We conclude that the monodromy action of $\pi_{1}(S^{*})$ on $H^{1}(X_{t}, \mathbb{Z})$ is given by 
$$T\mapsto \begin{pmatrix}1 & 3\\0 & 1\end{pmatrix}.$$


\end{example}
\begin{remark}
Every irreducible complex representation of an abelian group is $1$-dimensional, but if the group is not compact(finite group, or compact), then we don't necessarily have semi-simplicity.
\end{remark}



\end{document}