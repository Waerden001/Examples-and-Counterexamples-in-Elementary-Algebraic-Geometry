\documentclass[../main.tex]{subfiles}
\begin{document}
\subsection{Deformation theory}
\begin{example}[Hartsshorne $\mathrm{III}.9.9$, a rigid $k$-algebra]
Consider $$X=\mathrm{Spec}(k[x,y,z,w]/((x,y)\cap(z,w)))$$
\end{example}

\begin{remark}[infinitesimal deformation corresponding to $H^{1}(X,T_{X})$]
Thus we know projective spaces $\mathbb{P}_{k}^{n}$ are rigid. 
\end{remark}
\begin{example}[$T_{id}(\mathrm{Aut}(X))\cong \mathrm{Hom}(\Omega_{X}, \mathcal{O}_{X})$]
We assume $X$ is a scheme of finite type over $k=\overline{k}, char(k)=0$. A tangent vector of $\mathrm{Aut}(X)$ at the identity map is given by a morphism $$\mathrm{Spec}(k[\epsilon]/(\epsilon^{2}))\rightarrow \mathrm{Aut}(X).$$
In other words, an isomorphism over $k[\epsilon]/(\epsilon^{2})$
$$\mathcal{O}_{X[\epsilon]}\rightarrow \mathcal{O}_{X[\epsilon]}.$$
Locally, this is given by a $k[\epsilon]/(\epsilon^{2})$-algbra isomorphism
$$x_{i}\mapsto x_{i}+\epsilon f_{i}.$$
This is the same as given a morphism $d: \mathcal{O}_{X}\rightarrow \mathcal{O}_{X}$ such that 
$$d(x_{i}x_{j})=x_{i}f_{j}+x_{j}f_{i}=x_{i}d(f_{j})+x_{j}d(x_{i}).$$
In other words, a tangent vector is the same as a derivative $d: \mathcal{O}_{X}\rightarrow \mathcal{O}_{X}$. By the universal property of $\Omega_{X}$, we get an isomorphism 
$$T_{id}(\mathrm{Aut}(X))\cong \mathrm{Hom}(\Omega_{X}, \mathcal{O}_{X}).$$
If $X$ is smooth, we have 
$$T_{id}(\mathrm{Aut}(X))\cong H^{0}(X, T_{X}).$$

\end{example}

Let $X$ be a $k$-scheme. For a line bundle $\L\in \pic(X)$, we define the deformation functor 
$$\widehat{\pic}_{X, \L}:\cat{Art}/k\rightarrow \cat{Set}, \hspace{1cm} A\mapsto \{\text{line bundle } \L' \text{ on } X_{A}, \phi: \L'|_{X}\cong \L \}/\sim.$$
To get the tangent-obstruction theory of $\widehat{\pic}_{X, \L}$, we consider a small extension $0\rightarrow M\rightarrow B\rightarrow A\rightarrow 0$ in $\cat{Art}/k$. The map $\iota: X_{A}\hookrightarrow X_{B}$ is given by the ideal sheaf $\I=\pi^{*}M$, where $\pi: X_{B}\rightarrow \Spec(B)$ is the natural projection. We have an exact sequence of sheaves
$$1\rightarrow 1+\I\rightarrow \O^{\times}_{X_{B}}\rightarrow \O^{\times}_{X_{A}}\rightarrow 1.$$
Note that $1+\I\cong \I=\pi^{*}M$. Take the associated long exact sequence, we have 
$$\H^{1}(X, \pi^{*}M)\rightarrow \pic(X_{B})\xrightarrow{\iota^{*}}\pic(X_{A})\rightarrow \H^{2}(X, \pi^{*}M).$$ Note that locally the sheaf structure of $\pi^{*}M$ is just $\O_{X}\otimes_{k} M\cong \O_{X}^{\oplus \dim_{k}M}$. Thus we know
$$\H^{i}(X, \pi^{*}M)=\H^{i}(X, \O_{X})^{\oplus \dim_{k}M}=\H^{i}(X, \O_{X})\otimes_{k} M.$$
Assume that $X$ is proper. Then $T_{1}=\H^{1}(X, \O_{X})$ and $T_{2}=\H^{2}(X, \O_{X})$ are finite-dimension and form a tangent-obstruction theory. If $\pic(X)$ is representable, then we know the tangent space $T_{[\L]}\pic(X)=\H^{1}(X, \O_{X}).$
\subsection{Deformation of mapping spaces}
\subsection{Deformation of automorphisms}
\subsection{Deformation of torsors}
\subsection{Deformation of coherent sheaves}
Let $X$ be a smooth projective variety. $E$ be a coherent sheaf of $\O_{X}$-module. We denote the ring of dual numbers by $k[\epsilon]$, $\Spec(k[\epsilon])$ by $D$ and $X\otimes_{k}\Spec(k[\epsilon])$ by $X_{D}$. Consider the deformation functor 
$$ \M: \cat{Art}/k \rightarrow \cat{Set},\hspace{1cm}A \mapsto \{A\text{-flat coherent sheaf } E'\in \cat{Coh}(X_{A}), \phi:E'|_{X}\cong E\}/\sim.
$$
\begin{theorem}
There's a natural bijection between $\M(D)$ and $\ext^{1}(E, E)$. 
\end{theorem}
\begin{proof}
We have an short exact sequence 
$$0\rightarrow (\epsilon)\rightarrow k[\epsilon]\rightarrow k\rightarrow 0.$$
Note that as $k[\epsilon]$-modules, $(\epsilon)\cong k$. Pull the sequence back along the map $\pi:X_{D}\rightarrow D$, we get an exact sequence of $\O_{X_{D}}$-modules 
$$0\rightarrow \epsilon\O_{X}\rightarrow \O_{X_{D}}\rightarrow \O_{X}\rightarrow 0.$$
Let $E'$ be a deformation of $E$ over $D$. Since $E'$ is $D$-flat, applying $-\otimes E$ to the second sequence above, we get 
$$0\rightarrow \epsilon E\rightarrow E'\rightarrow E\rightarrow 0.$$
This gives us an element in $\ext^{1}_{\O_{X_{D}}}(\epsilon E, E)$, via the embedding $\O_{X}\rightarrow \pi_{*}\O_{X_{D}}=\O_{X}\oplus \epsilon\O_{X}$, it also gives us an element in $\ext^{1}_{\O_{X}}(E,E)$. Conversely, given an element $E'\in \ext^{1}_{\O_{X}}(E, E)$, we have to give $E'$ an $\O_{X_{D}}$-module structure and prove it's $D$-flat. For the first requirement, we just define the action of $\epsilon$ on $E'$ by the composition of $E'\rightarrow E$ and $E\rightarrow E'$, clearly we have $\epsilon^{2}=0$. To prove it's flat over $D$, we simply point out the $E'|_{X}=E$ which is flat over $\Spec(k)$ and $(\epsilon)\otimes_{\O_{X_{D}}} E'=E\rightarrow E'$ is an injection by our difinition of the $\epsilon$-action. Finally it's straightforward to check that 
$$\M(D)=\ext^{1}(E, E).$$
\end{proof}
\begin{theorem}
Let $$0\rightarrow I\rightarrow B\xrightarrow{\sigma}A\rightarrow 0$$
be a small extension and $E$ be a coherent sheaf on $X$ satisfying $\hom(E,E)\cong \C$(e.g stable sheaves). Then we have a tangent-obstruction theory given by 
$$\ext^{1}_{\O_{X}}(E, E)\otimes I\rightarrow \M(B)\rightarrow \M(A)\xrightarrow{ob} \ext^{2}_{\O_{X}}(E, E)\otimes I .$$
In other words, 
 \begin{enumerate}
   \item The non-trivial fibre of $\M(\sigma)$ is an $\ext^{1}_{\O_{X}}(E, E)\otimes_{k}I$-torsor.
   \item There is an obstruction map $\mathrm{ob}:\M(A)\rightarrow \ext^{2}_{\O_{X}}(E, E)\otimes_{k}I$. The image of $\M(\sigma)$ is exactly the kernel of the obstruction map. Furthermore,
   \item The image of $\mathrm{ob}$ is in the traceless part $\ext^{2}_{0}(E, E)\otimes_{k}I$, which is the kernel of the trace map $\ext^{1}(E, E)\otimes I\rightarrow \H^{2}(\O_{X})\otimes_{k} I$.  
 \end{enumerate}
\end{theorem}
\begin{proof}
We list the steps, when we have time, we'll write down the proof.
\begin{itemize}
    \item We first show that given $E_{A}\in \M(A)$. Deformations of $E_{A}$ to $B$ form a $\ext^{1}_{\O_{X_{A}}}(I\otimes_{\O_{X_{A}}}E_{A}, E_{A})$-torsor.
    \item Then we use some homological algebra to construct the obstruction map. 
    \item Finally,we apply some version of the cohomology and base change theorem to prove that $$\ext^{1}_{\O_{X_{A}}}(I\otimes_{\O_{X_{A}}}E_{A}, E_{A})=\ext^{1}_{\O_{X}}(E,E)\otimes_{k}I.$$
\end{itemize}
 
\end{proof}

\begin{remark}
From this example, we can actually see a deformation-obstruction theory has several layers. First, for any given geometric object $E$, we might discuss about its deformations even without a moduli space.  Secondly, with a fine moduli space, then we know $T_{[E]}M= \mathcal{M}(\Spec(k[\epsilon]/(\epsilon^{2})))$, where $\mathcal{M}$ is the moduli functor and $M$ is the moduli space. Thirdly, if we only have a coarse moduli space, then $T_{[E]}M\neq \mathcal{M}(\Spec(k[\epsilon]/(\epsilon^{2})))$ in general. Then you have to be careful what do you mean by `tangent space'.
\end{remark}


\subsection{Deformation of quotient sheaves}
Let $X$ be a scheme over $k$. $\F$ be a coherent sheaf on $X$, $\S_{0}$ be a coherent subsheaf of $\F$, $\qQ_{0}$ be the quotient. Define the deformation functor
$$D:=D_{[\F\rightarrow \qQ_{0}\rightarrow 0]}: \cat{Art}/k\rightarrow \cat{Set}$$
$$A\mapsto \{ \text{exact sequences of } A\text{-flat sheaves } 0\rightarrow \S\rightarrow \F\otimes_{k}A\rightarrow \qQ\rightarrow 0 \text{ on } X_{A},  \S\otimes_{A}k\cong \S_{0} \}/\sim.$$
By definition, $[0\rightarrow \S'\rightarrow \F\otimes_{k}B\rightarrow \qQ'\rightarrow 0]\in D(B)$ is a deformation of $[0\rightarrow \S\rightarrow \F\otimes_{k}A\rightarrow \qQ\rightarrow 0]\in D(A)$ if and only if we have 
\begin{itemize}
\item $\S'\otimes_{B}A=\S$
\item $\S'$ is flat over $B$. We know $\S'\otimes_{B}A=\S$ is flat over $A$, thus we only need $\S'\otimes_{B} M=\S\otimes_{B}A\otimes_{A}M=\S\otimes_{A}M \rightarrow \S'$ is an injection. In other words, flatness is equivalent to the existence of the short exact sequence
$$0\rightarrow \S\otimes_{A}M\rightarrow \S'\rightarrow \S\rightarrow 0.$$
\end{itemize}
In other words, an extension exists if and only if we can find a subsheaf $\E$ of  $\F\otimes_{k}B/S\otimes_{A}M=\F\otimes_{k}B/\im(\alpha)$ whose image in $\F\otimes_{k}A$ is exactly $\S$. Because the image lies in $\S$, we only need to find a subsheaf of $\ker(\beta)/\im(\alpha)$ has this property. In other words, we need the following sequence to split
$$0\rightarrow \ker\rightarrow \ker(\beta)/\im(\alpha)\rightarrow \S\rightarrow 0.$$
We can check that $\ker=\F\otimes_{k}M/\S\otimes_{A}M=\qQ\otimes_{A}M$. That is, the obstruction is given by 
$$ob=[0\rightarrow \qQ\otimes_{A}M\rightarrow \ker(\beta)/\im(\alpha)\rightarrow \S\rightarrow 0]\in \ext^{1}_{\O_{X_{B}}}(\S, \qQ\otimes_{A}M).$$
Moreover, all flat deformations of the given element is bijective to 
$$\ext^{0}_{\O_{X_{B}}}(\S, \qQ\otimes_{A}M)=\hom(\S, \qQ\otimes_{A}M).$$ 
$$\begin{tikzcd}
& 0\arrow{d} & & 0 \arrow{d}\\
& \S\otimes_{A}M\arrow{d}\arrow[dashrightarrow]{dr}{\alpha}& & \S\arrow{d}\\ 
0\arrow{r} & \F\otimes_{k}M\arrow{r}\arrow{d}& \F\otimes_{k}B\arrow[dashrightarrow]{dr}{\beta}\arrow{r}& \F\otimes_{k}A\arrow{r}\arrow{d} & 0\\
&\qQ\otimes_{A}M\arrow{d} & &Q\arrow{d}\\
& 0 & &0
\end{tikzcd}
$$
There's still one more issue, these two $\ext$ groups still depend on $B$. We claim that  the $M(\ker(\beta)/\im(\alpha))=0$. Indeed if $\gamma=\sum f_{i}\otimes_{k}b_{i}\in \ker(\beta)$, then we have $\sum f_{i}\otimes_{k}a_{i}\in \S$, where $a_{i}=b_{i}\pmod M$. Then $m\gamma =\sum f_{i}\otimes mb_{i}=\sum f_{i}\otimes ma_{i}$, the last identity comes from the fact that $M^{2}=0$, thus $ma_{i}$ is well defined. Then $m\gamma=m(\sum f_{i}\otimes_{k}a_{i})\in \im(\alpha)$. 
Apply $\otimes_{A}k$ to 
$$0\rightarrow \ker\rightarrow \ker(\beta)/\im(\alpha)\rightarrow \S\rightarrow 0.$$
We still get an exact sequence by the $A$-flatness of $\S$
$$0\rightarrow \qQ\otimes_{A}M\rightarrow \ker(\beta)/\im(\alpha)\otimes_{A}k\rightarrow \S\otimes_{A}k\rightarrow 0.$$
We can check that we have the following commutative diagram. If we just consider small extensions, we have $\qQ\otimes_{A}M\otimes_{A}k=\qQ\otimes_{A}M$ because $m_{A}M=0$.
$$
\begin{tikzcd}
& & 0\arrow{d}& 0\arrow{d}&\\
 & & \S\otimes_{A}m_{A}\arrow{d}\arrow[r, equal]&  \S\otimes_{A}m_{A}\arrow{d}&\\
 0\arrow{r}&\qQ\otimes_{A}M \arrow{r}\arrow[d,equal]& \ker(\beta)/\im(\alpha)\arrow{r}\arrow{d}&  \S\arrow{d}\arrow{r}& 0\\
 0\arrow{r}&\qQ\otimes_{A}M \arrow{r}& \ker(\beta)/\im(\alpha)\otimes_{A}k\arrow{r}\arrow{d}&\S\otimes_{A}k \arrow{d}\arrow{r}&0\\
 & & 0& 0& \\
\end{tikzcd}
$$
In conclusion, we have
$$\hom_{X_{A}}(\S,\qQ\otimes_{A}M)=\hom_{X}(\S\otimes_{A}k,\qQ\otimes_{A}M)=\hom_{X}(\S_{0}, \qQ_{0})\otimes_{k}M$$
$$\ext^{1}_{X_{A}}(\S, \qQ\otimes_{A}M)=\ext^{1}_{X}(\S\otimes_{A}k, \qQ\otimes_{A}M)=\ext^{1}_{X}(\S_{0}, \qQ_{0})\otimes_{k} M.$$
Because $m_{A}M=0$, any element in $\hom_{X_{A}}(\S, \qQ\otimes_{A}M)$ factors through the geometric fibre. And we can take out $M$ because of the projection formula, $M$ is simply viewed as a vector space. As a special case, the tangent-obstruction theory of the Hilbert scheme functor $H_{Z,X}$ is given by $T_{1}=\hom_{\O_{X}}(I_{Z}, \O_{Z}), T_{2}=\ext^{1}_{\O_{X}}(I_{Z}, \O_{Z})$.
\begin{remark}
We actually have many slightly different ways to describe the tangent space at a point on a Hilbert scheme of $n$ points on a $k$-dimensional smooth quasi-projective variety $X$.
\begin{align*}T_{Z}\mathrm{Hilb}^{n}(X)&=\hom_{\O_{X}}(\I_{Z}, \O_{Z})\\
&=\ext^1_{\O_{X}}(\O_{Z},\O_{Z})\\
&=\ext^{1}_{\O_{X}}(\I_{Z}, \I_{Z}).
\end{align*}
$\hom_{\O_{X}}(\I_{Z}, \O_{Z})=\ext^1_{\O_{X}}(\O_{Z},\O_{Z})$ can be seen by applying $\hom(-, \O_{Z})$ to the short exact sequence 
$$0\rightarrow \I_{Z}\rightarrow \O_{X}\rightarrow \O_{Z}\rightarrow 0.$$
For the last identity, maybe there's a much easier proof. The one we know is that we realize $\mathrm{Hilb}^{n}(X)$ as the rank $1$ instanton moduli space $\M(n,1)$(see the subsection about the deformation theory of $\M(n,r)$). The tangent space of $\M(n,1)$ at $[\I_{Z}]$ is given by $\ext^{1}_{\O_{X}}(I_{Z}, I_{Z})$. 
\end{remark}


\subsection{Deformation theory the instanton moduli space}
The instanton moduli space $\M(n,r)$ is defined to be moduli of rank $2$ torsion-free sheaves on $\P^{2}$ framed at infinity.
\subsection{Deformation of Hilbert schemes}
\subsection{Hilbert schemes of points on surfaces}
\subsection{Deformation of Calabi-Yau varieties}

\subsection{Deformation of varieties}
\subsection{Deformation of nodes}

\end{document}